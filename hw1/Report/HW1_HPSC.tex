\documentclass{article}
\usepackage[utf8]{inputenc}

\title{High Performance Scientific Computing \\ Homework Assignment 1}
% \subtitle{Homework Assignment 1}
\author{Naveen Himthani (120010001)}
\date{Spring 2015-16}

\usepackage{setspace}
\usepackage{amsmath,amsfonts,amssymb,amscd,amsthm,xspace}
\usepackage{titlesec}
\usepackage{vmargin}
\usepackage{fancyhdr}
\usepackage{caption}
\usepackage{subcaption}
\usepackage{multirow}
\usepackage{multicol}
\usepackage{url}
\usepackage{tabularx}
\usepackage{graphicx}
\usepackage{epstopdf}
\usepackage{booktabs}
\usepackage{rotating}
\usepackage{listings}
\usepackage{bm}
\usepackage{float}
\begin{document}


\maketitle

\section{Question 1}
Using Amadahl's law,
\begin{equation}
S(N) = \frac{1}{(1-P) + \frac{P}{N}}
\end{equation}
where for the given problem $(1-P)$ = 20$\%$ \\

\begin{enumerate}
\item Maximum speed-up achievable will be when $N=\infty$, therefore  \\
\begin{align}
S = \frac{1}{0.2 + 0} \\
\bf{S = 5}
\end{align}
\item For a desirable speed-up of $S=50$, the maximum serial portion of the algorithm can again be calculated when $N=\infty$ \\
\begin{align}
50 = \frac{1}{(1-P) + 0} \\
1-P = 0.02
\end{align}
therefore, the \textbf{serial portion of the algorithm is \bf{$2\%$}}

\section{Question 2}
Assumption $\Rightarrow$ sizeof(int) = 4 bytes
Total size of the data in main memory = $\frac{1024\times1024\times4}{1024} =  4096$ Kb
\subsection{CPU cycles in C program}
Since data is stored in the row-major format in C and the algorithm is written such that for each row a column is scanned, the data is read in a sequential manner without having to fetch large chunks of data for reading the next element as the algorithm requires.  
\begin{enumerate}
\item Cost of fetching 4096 Kb from main memory to cache = $4096\times150 = 614400$ CPU Cycles
\item Cost of fetching 1 Kb from Cache to CPU = $\frac{1024\times1}{8} = 128$ CPU Cycles
\item Therefore total cost of the algorithm = $614400+ 128\times4096 = 1138688$ CPU Cycles 
\end{enumerate}

\subsection{CPU cycles in FORTRAN program}
Since data is stored in the column-major format in FORTRAN and the algorithm is written such that for each row a column is scanned, the data is not read in a sequential manner i.e. it has to unnecessarily fetch large chunks of data for reading next element.
\begin{enumerate}
\item For accessing each element from the array in the order required by the algorithm, the cache will have to access 1 Kb data (256 integers) from main memory at a time which will incur 150 CPU cycles. Out of those 256 integers, the CPU only needs the first element every time for which it will access the first 2 elements (8 bytes) incurring 1 CPU cycle. In short, for getting 1 integer from the main memory to CPU costs $150+1=151$ CPU cycles.
\item This has to be done for all $1024\times1024$ elements of the array incurring $1024\times1024\times151 = 158334976$ CPU Cycles.
\end{enumerate}  





\end{document}
